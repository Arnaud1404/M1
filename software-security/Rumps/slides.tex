\documentclass[aspectratio=169]{beamer}

% --- THEME SETUP ---
\usetheme{Madrid}
% \usecolortheme{beaver} % Red/Grey theme for "Top Secret" vibe
\setbeamertemplate{navigation symbols}{} % Hide navigation commands (icons)

% --- CUSTOM WHITE BANNER LOGO SETUP ---
\setbeamercolor{frametitle}{bg=white, fg=blue!70!black}
\setbeamertemplate{frametitle}{%
    \vspace{0.2cm}
    \begin{beamercolorbox}[wd=\paperwidth, ht=0.7cm, dp=0.2cm]{frametitle}
        \hspace*{0.5cm}\usebeamerfont{frametitle}\insertframetitle\hfill
        \raisebox{-0.3cm}{\includegraphics[height=1cm]{logo_univ_bdx.png}}\hspace*{0.5cm}
    \end{beamercolorbox}%
    \begin{tikzpicture}[remember picture,overlay]
        \draw[blue!70!black, thick] (current page.north west) ++(0,-1.3cm) -- ++(\paperwidth,0);
    \end{tikzpicture}
}

% --- PACKAGES ---
\usepackage[utf8]{inputenc}     
\usepackage[T1]{fontenc}
\usepackage[french]{babel}
\usepackage{booktabs}
\usepackage{amsmath, amssymb}
\usepackage{tikz}
\usetikzlibrary{shapes.geometric, arrows.meta, positioning, calc, shadows}
\usepackage{pgfpages} % For speaker notes

% --- SPEAKER NOTES CONFIG ---
% Enable dual screen: Slides on Left, Notes on Right
\setbeameroption{show notes on second screen=right}
\setbeamertemplate{note page}[plain]

% --- LOGO CONFIG (MOVED TO FRAMETITLE) ---

% --- METADATA ---
\title{Operation Rubicon}
\subtitle{Analyse Cryptographique des Machines Crypto AG}
\author{Arnaud Gomes}
\institute{Université de Bordeaux}
\date{\today}

\begin{document}

% ==============================================================================
% 1. TITLE
% ==============================================================================
\begin{frame}
    \titlepage
    \note{
      \begin{itemize}
        \item \textbf{Sujet :} Plus grande opération d'espionnage crypto du XXe siècle.
        \item \textbf{Acteurs :} CIA + BND possèdent secrètement Crypto AG (leader mondial chiffrement) pendant 50~ans.
        \item \textbf{Cible :} Équipements vendus à diplomates et gouvernements du monde entier.
        \item \textbf{Nature de la faille :} Pas une erreur de code~; backdoor mathématique implantée au niveau matériel.
        \item \textbf{Objectif :} Affaiblir l'espace des clés de multiples gouvernements.
      \end{itemize}
    }
\end{frame}

% ==============================================================================
% 2. CONTEXT (MULTIPLE SLIDES)
% ==============================================================================
\section{Contexte Géopolitique}

\begin{frame}{Le Contexte : Le Partenariat CIA / BND}
    \vspace{-10pt}
    \begin{columns}
        \column{0.4\textwidth}
        \textbf{L'Opération Rubicon (Thesaurus)}\\
        \vspace{0.3cm}
        \begin{itemize}
            \item Accord secret signé en 1970.
            \item Achat de Crypto AG via des sociétés écrans au Liechtenstein.
            \item Partage à 50/50 des bénéfices... et des interceptions diplomatiques.
        \end{itemize}

        \column{0.6\textwidth}
        \begin{center}
            \resizebox{\linewidth}{!}{
            \begin{tikzpicture}[thick, >=Latex]
                \tikzset{
                    agencybox/.style={draw=black, text width=2.8cm, align=center, rounded corners, minimum height=2cm},
                    targetbox/.style={draw=black, text width=3.2cm, align=center, rounded corners, minimum height=2.4cm}
                }

                % --- Nodes (symmetrical Y-layout) ---
                \node[agencybox, fill=red!10] (cia) at (0,2) {
                    \includegraphics[height=1cm]{images/cia_seal.png}\\
                    \vspace{0.1cm}
                    \textbf{NSA / CIA}
                };
                \node[agencybox, fill=gray!20] (bnd) at (0,-2) {
                    \includegraphics[height=1cm]{images/bnd_hq.jpg}\\
                    \vspace{0.1cm}
                    \textbf{ZFA / BND}
                };
                \node[targetbox, fill=blue!10] (cag) at (7,0) {
                    \includegraphics[height=1cm]{images/crypto_ag_logo.png}\\
                    \vspace{0.1cm}
                    \textbf{Crypto AG}\\
                    \textit{Zoug, Suisse}
                };

                % --- Context labels to make the diagram self-contained ---
                % (Removed floating "Accord secret" label to avoid overlap)
                \node[font=\scriptsize, align=center, text=red, xshift=-20pt] at (4.1,0) {\textbf{Partage 50/50}\\bénéfices + interceptions};

                % --- Solid arrows: Agency -> Crypto AG (inner path) ---
                \draw[-{Latex[length=3mm]}] (cia.east) to[out=-8, in=158]
                    node[pos=0.52, above, sloped, font=\footnotesize] {Cahier des charges}
                    (cag.north west);
                \draw[-{Latex[length=3mm]}] (bnd.east) to[out=8, in=-158]
                    node[pos=0.52, below, sloped, font=\footnotesize] {Ingénierie}
                    (cag.south west);

                % --- Dashed red: Renseignements back to agencies (outer path) ---
                \draw[-{Latex[length=3mm]}, dashed, red] (cag.north) to[out=118, in=28]
                    node[pos=0.55, above, font=\footnotesize, text=red] {Renseignements}
                    (cia.east);
                \draw[-{Latex[length=3mm]}, dashed, red] (cag.south) to[out=-118, in=-28]
                    node[pos=0.55, below, font=\footnotesize, text=red] {Renseignements}
                    (bnd.east);
            \end{tikzpicture}
            }
        \end{center}
    \end{columns}

    \begin{tikzpicture}[remember picture,overlay]
        \node[anchor=south east, xshift=-0.5cm, yshift=0.5cm, font=\tiny, text=gray!50] at (current page.south east) {[Vidéo : Sylvqin, L'Affaire Crypto AG, 2021]};
    \end{tikzpicture}

    \note{
      \begin{itemize}
        \item \textbf{Date clé :} 1970, Guerre Froide~; accord secret ``Thesaurus'' (rebaptisé Rubicon).
        \item \textbf{Montage :} Rachat de Crypto AG (Zoug, Suisse) via sociétés écrans au Liechtenstein.
        \item \textbf{Partage :} Bénéfices commerciaux 50/50 entre CIA et BND.
        \item \textbf{Point crucial :} NSA et ZFA imposent les algorithmes de chiffrement~; contrôle total de la chaîne de production.
      \end{itemize}
    }
\end{frame}

\begin{frame}{Le Contexte : "Trusted" Hardware}
    \begin{columns}
        \column{0.5\textwidth}
        \begin{alertblock}{Le Couvert de la Neutralité Suisse}
            Leader mondial du chiffrement matériel (ex: HC-520, HC-570).
            Vendu à +120 pays sous couvert de stricte neutralité.
        \end{alertblock}

        \vspace{0.2cm}
        \textbf{Le Mécanisme Commercial :}
        \begin{itemize}
            \item \textbf{Machines Dédiées :} Boîtiers électromécaniques lourds.
            \item \textbf{Boîte Noire :} Algorithmes propriétaires hardware non-documentés ("Security by Obscurity").
            \item \textbf{Légitimité :} Promesse de sécurité mathématique par Boris Hagelin.
        \end{itemize}

        \column{0.5\textwidth}
        \begin{center}
            \includegraphics[height=3.5cm]{images/hagelin.jpg}\\
            \vspace{0.1cm}
            \tiny \textit{Une machine de type Hagelin CX-52.}
        \end{center}
    \end{columns}

    \note{
      \begin{itemize}
        \item \textbf{Neutralité suisse :} Aura politique $\Rightarrow$ fournisseur idéal pour tous les gouvernements.
        \item \textbf{Hardware lourd :} Blocs physiques (Hagelin) considérés inviolables.
        \item \textbf{Boîte noire :} Algorithmes HC-500 gravés dans le silicium, aucune doc mathématique fournie.
        \item \textbf{Danger :} ``Security by Obscurity'' $=$ concept fondamentalement vicié en cryptographie.
      \end{itemize}
    }
\end{frame}

\begin{frame}{Le Contexte : Les Deux Versions (A et B)}
    \begin{columns}
        \column{0.5\textwidth}
        \textbf{Version A : "Alliés"}
        \begin{itemize}
            \item États-Unis, Royaume-Uni, OTAN.
            \item Machines totalement sécurisées.
            \item Algorithme robuste non-compromis.
        \end{itemize}

        \column{0.5\textwidth}
        \textbf{Version B : "Le Reste du Monde"}
        \begin{itemize}
            \item Iran, Libye, Argentine, Inde, Vatican...
            \item Machines comportant la backdoor implantée par la NSA.
            \item Messages lisibles en temps réel par NSA/BND.
        \end{itemize}
    \end{columns}
    
    \vspace{0.8cm}
    \begin{center}
        \textit{Même les ingénieurs et commerciaux de Crypto AG (ex: Hans Bühler en Iran) ignoraient manipuler des versions truquées.}
    \end{center}

    \note{
      \begin{itemize}
        \item \textbf{Double production :} Même machine, algorithme interne différent.
        \item \textbf{Version A :} OTAN / alliés~; chiffrement robuste non-compromis.
        \item \textbf{Version B :} Reste du monde (Iran, Libye, Argentine, Inde\ldots)~; backdoor NSA implantée.
        \item \textbf{Résultat :} Câbles diplomatiques lisibles en clair par NSA/BND.
        \item \textbf{Cloisonnement :} Même les ingénieurs Crypto AG ignoraient la fraude~; seule une micro-cellule concevait la faille.
      \end{itemize}
    }
\end{frame}

% ==============================================================================
% 2.5 HISTORICAL IMPACT SLIDES
% ==============================================================================

\begin{frame}{L'Impact Historique : Exploitation des failles}
    \textbf{Cas n°1 : La Guerre des Malouines (1982)}

    \vspace{0.3cm}
    \begin{itemize}
        \item \textbf{Le Contexte :} Conflit armé entre le Royaume-Uni et l'Argentine (cliente de Crypto AG, ``Version B'').
        \item \textbf{L'Exploitation :} La junte militaire argentine chiffrait l'intégralité de ses communications navales tactiques avec des machines de la série Hagelin CX-52 / HC-500.
        \item \textbf{Résultat Opérationnel :} La NSA déchiffre les positions navales argentines en temps réel et transmet les renseignements à Londres via les accords Five Eyes.
    \end{itemize}

    \vspace{0.5cm}
    \begin{columns}
        \column{0.8\textwidth}
        \begin{alertblock}{Asymétrie du Renseignement}
            L'Argentine, pensant son canal diplomatique sécurisé, négociait publiquement aux Nations-Unies tout en planifiant des frappes. Le Royaume-Uni, bénéficiaire des interceptions NSA/BND, disposait d'un avantage informationnel décisif.
        \end{alertblock}
    \end{columns}

    \note{
      \begin{itemize}
        \item \textbf{Cas concret :} Guerre des Malouines, 1982.
        \item \textbf{Confiance aveugle :} Argentine chiffrait tout via Hagelin CX-52/HC-500 (Version B).
        \item \textbf{Compromission totale :} Canaux tactiques et stratégiques lisibles par la NSA.
        \item \textbf{Relais Five Eyes :} NSA déchiffre en temps réel $\Rightarrow$ renseignements transmis à Londres.
        \item \textbf{Avantage décisif :} UK connaissait positions navales argentines avant les frappes.
      \end{itemize}
    }
\end{frame}

\begin{frame}{L'Impact Historique : L'Iran et la Libye}
    \textbf{Cas n°2 : Espionnage étatique et Anti-terrorisme (Années 70-80)}

    \vspace{0.3cm}
    \begin{columns}
        \column{0.5\textwidth}
        \textbf{Crise des otages en Iran (1979)}
        \begin{itemize}
            \item Crise diplomatique : 52 américains retenus à Téhéran.
            \item Jimmy Carter (USA) observe la diplomatie ennemie en temps réel via l'interception des HC-500 iraniennes.
        \end{itemize}

        \column{0.5\textwidth}
        \textbf{Attentat "La Belle" Berlin (1986)}
        \begin{itemize}
            \item Ronald Reagan accuse Mouammar Kadhafi de l'attentat de Berlin-Ouest.
            \item Preuve formelle~: les télégrammes de ``félicitations'' libyens chiffrés par Crypto AG ont été déchiffrés quasi-instantanément par l'infrastructure de la NSA, fournissant un accès direct au texte clair.
        \end{itemize}
    \end{columns}

    \vspace{0.6cm}
    \textit{Conséquence globale~: durant toute la Guerre Froide, la CIA a intercepté les communications de plus de 120~pays de manière systématique.}

    \begin{tikzpicture}[remember picture,overlay]
        \node[anchor=south east, xshift=-0.5cm, yshift=0.5cm, font=\tiny, text=gray!50] at (current page.south east) {[Vidéo : Sylvqin, L'Affaire Crypto AG, 2021]};
    \end{tikzpicture}

    \note{
      \begin{itemize}
        \item \textbf{Iran 1979 :} HC-500 Version B léguées par le Shah~; Carter lit les discussions internes de Khomeini en temps réel.
        \item \textbf{Berlin 1986 :} Attentat ``La Belle''~; Reagan invoque des ``preuves absolues''.
        \item \textbf{Source secrète :} Câbles libyens (``Félicitations pour Berlin'') déchiffrés quasi-instantanément via backdoor Crypto AG.
        \item \textbf{Bilan global :} Trafic diplomatique du bloc non-aligné systématiquement intercepté~; +120~pays compromis.
      \end{itemize}
    }
\end{frame}

% ==============================================================================
% 3. TECHNICAL ARCHITECTURE 
% ==============================================================================
\section{Architecture Technique}

\begin{frame}{Architecture : Chiffrement par flot}
    \textbf{Modèle (Cours Chap. II, Sect. 3)}

    \vspace{0.3cm}
    Les machines Crypto AG utilisent un \textbf{chiffrement par flot synchrone}.
    Le message clair $m_t$ est chiffré bit à bit avec la suite chiffrante $z_t$ :
    
    \begin{block}{Équation Fondamentale}
    $$c_t = m_t \oplus z_t$$
    \end{block}

    \vspace{0.3cm}
    \textbf{Pourquoi en hardware (1970--1990) ?}
    \begin{itemize}
        \item Implémentation en portes logiques extrêmement compacte.
        \item Symétrie : déchiffrement identique ($m_t = c_t \oplus z_t$).
        \item Pas de propagation d'erreur sur ligne radio/télex.
    \end{itemize}

    \note{
      \begin{itemize}
        \item \textbf{Réf. cours :} Chap.~II, Sect.~3~: chiffrement par flot synchrone.
        \item \textbf{XOR :} Bit à bit $\Rightarrow$ trivial en silicium.
        \item \textbf{Tolérance erreur :} 1~bit corrompu $=$ 1~bit perdu (pas de propagation télex).
        \item \textbf{Clé de voûte :} Sécurité $\equiv$ qualité de la suite chiffrante $z_t$.
      \end{itemize}
    }
\end{frame}

% --- SLIDE 1: RAW LFSR ---
\begin{frame}{Le Générateur : LFSR}
    La suite chiffrante est produite par un \textbf{LFSR} (Linear Feedback Shift Register) : un automate linéaire sur $\mathbb{F}_2$.
    
    \begin{center}
    \resizebox{0.52\linewidth}{!}{
    \begin{tikzpicture}[thick, scale=0.85, every node/.style={transform shape}]
        \tikzset{
            regcell/.style={draw, minimum width=1.2cm, minimum height=0.55cm},
            dotcell/.style={draw, minimum width=0.55cm, minimum height=0.55cm},
            multiplier/.style={draw, circle, inner sep=0pt, minimum size=0.4cm, font=\scriptsize},
            xor/.style={inner sep=0pt, minimum size=0.35cm, font=\large}
        }

        % Register Cells
        \node[regcell] (sL-1) at (0,0) {\scriptsize $S_{\ell-1}^{(t)}$};
        \node[regcell, right=0pt of sL-1] (sL-2) {\scriptsize $S_{\ell-2}^{(t)}$};
        \node[dotcell, right=0pt of sL-2] (dots1) {\scriptsize $\dots$};
        \node[dotcell, right=0pt of dots1] (dots2) {\scriptsize $\dots$};
        \node[dotcell, right=0pt of dots2] (dots3) {\scriptsize $\dots$};
        \node[regcell, right=0pt of dots3] (s1) {\scriptsize $S_{1}^{(t)}$};
        \node[regcell, right=0pt of s1] (s0) {\scriptsize $S_{0}^{(t)}$};

        % Multipliers
        \node[multiplier, below=0.2cm of sL-1] (c1) {$c_1$};
        \node[multiplier, below=0.2cm of sL-2] (c2) {$c_2$};
        \node[multiplier, below=0.2cm of s1] (cL-1) {\tiny $c_{\ell\text{-}1}$};
        \node[multiplier, below=0.2cm of s0] (cL) {$c_\ell$};

        % XOR Gates
        \node[xor, below=0.2cm of c1] (x1) {$\bigoplus$};
        \node[xor, at={(c2 |- x1)}] (x2) {$\bigoplus$};
        \node[xor, at={(cL-1 |- x1)}] (xL-1) {$\bigoplus$};

        \draw[dotted, thick] (x2) -- (xL-1);

        % Connections
        \draw[-{Latex[length=1.5mm]}] (sL-1) -- (c1);
        \draw[-{Latex[length=1.5mm]}] (sL-2) -- (c2);
        \draw[-{Latex[length=1.5mm]}] (s1) -- (cL-1);
        \draw[-{Latex[length=1.5mm]}] (s0) -- (cL);
        \draw[-{Latex[length=1.5mm]}] (c1) -- (x1);
        \draw[-{Latex[length=1.5mm]}] (c2) -- (x2);
        \draw[-{Latex[length=1.5mm]}] (cL-1) -- (xL-1);
        \draw[-] (cL) -- (cL |- x1);
        \draw[-{Latex[length=1.5mm]}] (cL |- x1) -- (xL-1);
        \draw[-{Latex[length=1.5mm]}] (xL-1) -- ++(-1.0,0);
        \draw[-{Latex[length=1.5mm]}] (x2) -- (x1);
        \draw[-{Latex[length=1.5mm]}] (x1) -- ++(-1.2,0) |- (sL-1.west);

        % OUTPUT FROM S0
        \draw[-{Latex[length=1.5mm]}, thick, blue!70!black] (s0.east) -- ++(1.0,0) node[right, font=\large] {$z_t$};
    \end{tikzpicture}
    }
    \end{center}

    \begin{itemize}
        \item \textbf{État} : $S^{(t)} = (S_0^{(t)}, \dots, S_{\ell-1}^{(t)}) \in \mathbb{F}_2^{\ell}$ \hfill (Def. II-4)
        \item \textbf{Mise à jour} : Récurrence linéaire sur $\mathbb{F}_2$.
        \item \textbf{Polynôme de rétroaction} : $f(X) = 1 \oplus c_1 X \oplus \dots \oplus c_{\ell} X^{\ell}$.
        \item \textbf{Période} : $T = 2^{\ell} - 1$ (m-suite, si $f$ primitif). \hfill (Prop. II-5)
    \end{itemize}

    \note{
      \begin{itemize}
        \item \textbf{Def.~II-4 :} LFSR $=$ automate linéaire sur $\mathbb{F}_2$.
        \item \textbf{Prop.~II-5 :} Période max $\iff$ polynôme primitif.
        \item \textbf{Sortie :} $z_t = S_0^{(t)}$ (output direct du registre).
        \item \textbf{Limite :} Hardware efficace mais structure algébrique transparente.
      \end{itemize}
    }
\end{frame}

% --- SLIDE 2: VULNERABILITY ---
\begin{frame}{La Faiblesse : Berlekamp-Massey}
    \begin{alertblock}{La Linéarité tue la Sécurité}
        Un LFSR brut est régi par des équations linéaires sur $\mathbb{F}_2$. Son polynôme est reconstructible.
    \end{alertblock}

    \vspace{0.2cm}
    \textbf{L'Algorithme de Berlekamp-Massey :}
    \begin{itemize}
        \item Reconstruit le polynôme minimal $f(X)$ d'une suite récurrente linéaire.
        \item Complexité : $\mathcal{O}(\ell^2)$ opérations sur $\mathbb{F}_2$.
        \item Il suffit d'observer \textbf{$2\ell$ bits} consécutifs de la suite chiffrante $z_t$ pour retrouver $S^{(0)}$ et prédire toute la suite.
    \end{itemize}

    \vspace{0.2cm}
    \textit{Pour vendre une machine ``inviolable'', Crypto AG devait briser cette transparence algébrique.}

    \note{
      \begin{itemize}
        \item \textbf{Réf. cours :} Def.~II-8 et Prop.~II-9 (Berlekamp-Massey).
        \item \textbf{Coût :} $2\ell$ bits observés $\Rightarrow$ reconstruction complète de $f(X)$ et $S^{(0)}$.
        \item \textbf{Verdict :} LFSR brut $=$ sécurité nulle face à adversaire compétent.
        \item \textbf{Transition :} Nécessité d'introduire de la non-linéarité.
      \end{itemize}
    }
\end{frame}

\begin{frame}{La Solution : Générateur à Combinaison}
    \begin{columns}[c]
        \begin{column}{0.55\textwidth}
            \centering
            \resizebox{\linewidth}{!}{
            \begin{tikzpicture}[thick, scale=1.05, every node/.style={transform shape}]
                \tikzset{
                    lfsrbox/.style={draw, rounded corners, minimum height=0.8cm, font=\small, align=center, fill=gray!10},
                    gbox/.style={draw, rounded corners, fill=blue!10, minimum width=2.5cm, minimum height=5.0cm, font=\large, align=center, drop shadow={shadow xshift=0.8mm, shadow yshift=-0.8mm, opacity=0.4}}
                }

                \node[lfsrbox, minimum width=6.5cm] (lfsr1) at (0, 2.2)  {$LFSR_1$ \quad (longueur $\ell_1$)};
                \node[lfsrbox, minimum width=6.0cm] (lfsr2) at (0, 0.8)  {$LFSR_2$ \quad (longueur $\ell_2$)};
                \node at (0, -0.3) {\Huge $\vdots$};
                \node[lfsrbox, minimum width=5.0cm] (lfsrk) at (0, -1.4) {$LFSR_k$ \quad (longueur $\ell_k$)};

                \node[gbox] (gbox) at (7.5, 0.4) {$g$};
                \node[below=0.01cm of gbox, font=\scriptsize\itshape, text=blue!70!black] {Combinaison non-linéaire};

                \draw[-{Latex[length=3mm]}, thick] (lfsr1.east) -- node[above, font=\scriptsize, midway] {$x_1^{(t)}$} (lfsr1.east -| gbox.west);
                \draw[-{Latex[length=3mm]}, thick] (lfsr2.east) -- node[above, font=\scriptsize, midway] {$x_2^{(t)}$} (lfsr2.east -| gbox.west);
                \draw[-{Latex[length=3mm]}, thick] (lfsrk.east) -- node[above, font=\scriptsize, midway] {$x_k^{(t)}$} (lfsrk.east -| gbox.west);

                \draw[-{Latex[length=4mm]}, thick, red!70!black] (gbox.east) -- ++(1.5,0) node[right, font=\Huge] {$z_t$};
            \end{tikzpicture}
            }
        \end{column}

        \begin{column}{0.45\textwidth}
            On combine $k$ LFSRs \textbf{indépendants} via une fonction non-linéaire $g$.

            \vspace{0.4cm}
            \begin{block}{Propriétés du Générateur}
                \begin{itemize}
                    \item \textbf{Génération :} $z_t = g(x_1^{(t)}, \dots, x_k^{(t)})$
                    \item \textbf{Période Maximale :} Les $\ell_i$ sont \textbf{premières entre elles}.
                    \item \textbf{Résultat :} $T_{tot} = \prod_{i=1}^k (2^{\ell_i} - 1)$.
                \end{itemize}
            \end{block}
        \end{column}
    \end{columns}

    \begin{tikzpicture}[remember picture,overlay]
        \node[anchor=south east, xshift=-0.5cm, yshift=0.5cm, font=\tiny, text=gray!50] at (current page.south east) {[Modèle : Pornin, SSTIC]};
    \end{tikzpicture}

    \note{
      \begin{itemize}
        \item \textbf{Industrie :} $k$ entre 3 et 6 LFSRs~; ASIC compacts.
        \item \textbf{Longueurs $\ell_i$ :} Premières entre elles $\Rightarrow$ période globale $= \prod(2^{\ell_i}-1)$, gigantesque.
        \item \textbf{Rôle de $g$ :} Fonction non-linéaire censée rendre Berlekamp-Massey incalculable.
        \item \textbf{Astuce NSA :} $g$ conçue pour paraître robuste, mais immunité de corrélation délibérément sacrifiée.
        \item \textbf{Conséquence :} Biais statistique $\epsilon$ introduit $\Rightarrow$ attaque par corrélation (Divide \& Conquer) rendue possible.
      \end{itemize}
    }
\end{frame}

\begin{frame}{La Backdoor Statistique de Crypto AG}
    La NSA conçoit en secret une fonction $g$ \textbf{biaisée} gravée en silicium.
    Le combineur échoue sciemment à satisfaire le critère d'\textbf{immunité de corrélation}.

    \vspace{0.3cm}
    \begin{alertblock}{Le Biais Exploitable}
        Il existe un registre $LFSR_1$ et un biais $\epsilon > 0$ tel que :
        $$P(z_t = x_1^{(t)}) = 0.5 + \epsilon$$
    \end{alertblock}

    \vspace{0.3cm}
    \textbf{Conséquence :} La suite chiffrante $z_t$ ``fuite'' de l'information sur la sortie $x_1^{(t)}$ d'un registre individuel.

    \vspace{0.2cm}
    \textit{Ce biais est invisible à l'usage quotidien, mais statistiquement exploitable avec suffisamment de chiffré.}

    \begin{tikzpicture}[remember picture,overlay]
        \node[anchor=south east, xshift=-0.5cm, yshift=0.5cm, font=\tiny, text=gray!50] at (current page.south east) {[Biais $\epsilon$ : CCC, Reverse-Engineering 2020]};
    \end{tikzpicture}

    \note{
      \begin{itemize}
        \item \textbf{Backdoor :} Violation délibérée de l'immunité de corrélation de $g$.
        \item \textbf{Façade :} $g$ suffisamment complexe pour résister à Berlekamp-Massey $\Rightarrow$ aspect ``sécurisé''.
        \item \textbf{Faille cachée :} Immunité de corrélation sacrifiée $\Rightarrow$ biais $\epsilon$ exploitable.
        \item \textbf{Stratégie NSA :} Faiblesse statistique dissimulée derrière la complexité apparente de $g$.
        \item \textbf{``Henry'' (Widman) :} Kjell-Ove Widman, mathématicien suédois recruté par la CIA (alias ``Henry''), justifiait académiquement les algorithmes affaiblis auprès des ingénieurs suisses~; le haut degré algébrique apparent de $g$ masquait la vulnérabilité statistique lors des revues internes.
      \end{itemize}
    }
\end{frame}

% ==============================================================================

% 5. ATTACK COMPLEXITY
% ==============================================================================
\section{L'Attaque (Divide \& Conquer)}

\begin{frame}{L'Attaque : \textit{Divide \& Conquer}}
    Le biais $\epsilon$ permet d'attaquer chaque registre $L_i$ \textbf{indépendamment}.

    \vspace{0.3cm}
    \begin{columns}
        \column{0.5\textwidth}
        \begin{center}
            \textbf{Brute Force (sans backdoor)}\\
            \vspace{0.2cm}
            \textcolor{red}{\Large $\mathcal{O}(2^{\sum_{i=1}^k \ell_i})$}\\
            \vspace{0.2cm}
            Espace joint\\
            \textit{Incalculable (siècles).}
        \end{center}

        \column{0.5\textwidth}
        \begin{center}
            \textbf{Corrélation (Backdoor)}\\
            \vspace{0.2cm}
            \textcolor{green!60!black}{\Large $\mathcal{O}(\sum_{i=1}^k 2^{\ell_i})$}\\
            \vspace{0.2cm}
            Registres isolés\\
            \textit{Quelques secondes.}
        \end{center}
    \end{columns}

    \vspace{0.4cm}
    \textbf{Méthode :} Avec du \textit{clair probable} (en-tête diplomatique, salutation standard), l'attaquant compare statistiquement chaque état candidat de $L_1$ à la suite chiffrante captée. Le vrai état se démarque par corrélation.

    \begin{tikzpicture}[remember picture,overlay]
        \node[anchor=south east, xshift=-0.5cm, yshift=0.5cm, font=\tiny, text=gray!50] at (current page.south east) {[KPA Methodology : Pornin, SSTIC]};
    \end{tikzpicture}

    \note{
      \begin{itemize}
        \item \textbf{Type :} Attaque à clair connu (KPA).
        \item \textbf{Source du clair :} En-têtes régulières télex diplomatiques (``To Mr Ambassador\ldots'').
        \item \textbf{Récupération :} $m_t \oplus c_t = z_t$ $\Rightarrow$ suite chiffrante extraite.
        \item \textbf{Cascade :} Isoler $LFSR_1$ via biais $\epsilon_1$, soustraire, attaquer $LFSR_2$, etc.
        \item \textbf{Gain :} Exponentielle de la somme $\to$ somme d'exponentielles.
      \end{itemize}
    }
\end{frame}


\begin{frame}[fragile]{Visualisation de l'Attaque en Cascade}
    \begin{center}
    \resizebox{0.85\linewidth}{!}{
    \begin{tikzpicture}[auto, thick, node distance=1.5cm and 1.5cm]
        \tikzset{
            stepnode/.style={draw, rounded corners, align=center, text width=4.2cm, minimum height=1.4cm, font=\small, text centered},
            capture/.style={stepnode, fill=red!10},
            attack/.style={stepnode, fill=yellow!20},
            remove/.style={stepnode, fill=green!20},
            finish/.style={stepnode, fill=blue!10, text width=6.5cm, font=\bfseries}
        }

        % --- ROW 1 ---
        \node[capture] (c) {1. Capture de Données\\($z_t = m_t \oplus c_t$)};
        \node[attack, right=of c] (a1) {2. Attaque sur $LFSR_1$\\\textcolor{red}{$\mathcal{O}(2^{\ell_1})$}};
        \node[remove, right=of a1] (r1) {3. Retrait influence\\ de $LFSR_1$};

        % --- ROW 2 ---
        \node[attack, below=1.8cm of c] (a2) {4. Attaque sur $LFSR_2$\\\textcolor{red}{$\mathcal{O}(2^{\ell_2})$}};
        \node[stepnode, fill=gray!5, right=of a2] (dots) {\Huge $\dots$};
        \node[attack, right=of dots] (ak) {5. Attaque finale sur $LFSR_k$\\\textcolor{red}{$\mathcal{O}(2^{\ell_k})$}};

        % --- ROW 3 ---
        \node[finish, below=1.5cm of dots] (f) {État Reconstruit $\implies$ Clair Identifié};

        % Arrows Row 1
        \draw[-{Latex[length=3mm]}] (c) -- (a1);
        \draw[-{Latex[length=3mm]}] (a1) -- (r1);

        % Serpent Connector Row 1 -> Row 2
        \draw[-{Latex[length=3mm]}, rounded corners=10pt] (r1.south) -- ++(0,-0.6) -| (a2.north);

        % Arrows Row 2
        \draw[-{Latex[length=3mm]}] (a2) -- (dots);
        \draw[-{Latex[length=3mm]}] (dots) -- (ak);

        % Serpent Connector Row 2 -> Row 3
        \draw[-{Latex[length=3mm]}, rounded corners=10pt] (ak.south) -- ++(0,-0.6) -| (f.north);

    \end{tikzpicture}
    }
    \end{center}

    \vspace{0.3cm}
    \textbf{Complexité Totale :} $\mathcal{O}(\sum_{i=1}^{k} 2^{\ell_i}) \ll \mathcal{O}(2^{\sum_{i=1}^{k} \ell_i})$

    \begin{tikzpicture}[remember picture,overlay]
        \node[anchor=south east, xshift=-0.5cm, yshift=0.5cm, font=\tiny, text=gray!50] at (current page.south east) {[Cascade Iteration : CCC, 2020]};
    \end{tikzpicture}

    \note{
      \begin{itemize}
        \item \textbf{Principe :} Brise l'exposant (somme des longueurs) en somme linéaire.
        \item \textbf{Exemple :} $k=6$ $\Rightarrow$ quelques secondes vs. millénaires en brute-force.
        \item \textbf{Signal purifié :} Chaque étape soustrait l'influence des registres déjà cassés.
      \end{itemize}
    }
\end{frame}

% ==============================================================================
% 6. LA CHUTE DE L'EMPIRE CRYPTO AG (HANS BÜHLER)
% ==============================================================================
\section{La Chute d'Operation Rubicon}

\begin{frame}{1992 : La Faille Humaine (L'Affaire Hans Bühler)}
    \begin{columns}
        \column{0.6\textwidth}
        \textbf{L'Arrestation en Iran}
        \begin{itemize}
            \item Hans Bühler, ingénieur commercial star de Crypto AG, est arrêté à Téhéran en \textbf{mars 1992}.
            \item Interrogé pendant \textbf{9~mois} par les services iraniens, libéré en \textbf{janvier 1993}.
            \item Le gouvernement iranien suspecte l'équipement d'être compromis suite à des fuites liées à des assassinats politiques.
        \end{itemize}

        \vspace{0.3cm}
        \textbf{L'Opération Démasquée}
        \begin{itemize}
            \item Le \textbf{BND} paie la rançon/caution de \textbf{1\,M\,\$} à l'Iran~; la \textbf{CIA refuse} de contribuer.
            \item Ce différend financier provoque le \textit{Divorce Agreement} de 1993~: la CIA rachète la part du BND pour \textbf{17\,M\,\$}.
            \item Crypto AG licencie Bühler~; l'attention médiatique fait s'écrouler le mythe de la ``neutralité suisse''.
        \end{itemize}

        \column{0.4\textwidth}
        \begin{center}
            \begin{alertblock}{Compétence vs.\ Backdoor}
                \small
                Bühler ignorait la fraude, \textbf{mais Menga Caflisch}, ingénieure brillante, a accidentellement \textbf{corrigé la backdoor} en optimisant les algorithmes~--- forçant la CIA à stopper la production.
            \end{alertblock}
        \end{center}
    \end{columns}

    \begin{tikzpicture}[remember picture,overlay]
        \node[anchor=south east, xshift=-0.5cm, yshift=0.5cm, font=\tiny, text=gray!50] at (current page.south east) {[Vidéo : Sylvqin, L'Affaire Crypto AG, 2021]};
    \end{tikzpicture}

    \note{
      \begin{itemize}
        \item \textbf{Ironie :} Pas un audit crypto qui révèle la faille, mais une erreur humaine.
        \item \textbf{Arrestation :} Mars 1992~; Hans Bühler arrêté à Téhéran~; Iran suspecte les machines.
        \item \textbf{Interrogatoire :} 9~mois de détention~; Bühler ignorait la fraude.
        \item \textbf{Rançon :} BND paie 1~M\$~; CIA refuse $\Rightarrow$ peur de griller la couverture.
        \item \textbf{Divorce 1993 :} CIA rachète part BND pour 17~M\$~; contrôle exclusif.
        \item \textbf{Fuite médiatique :} Bühler alerte la presse~; lien Zoug--BND découvert.
        \item \textbf{Menga Caflisch :} Ingénieure chez Crypto AG~; en optimisant les algorithmes (son travail normal), elle corrige accidentellement la backdoor NSA.
        \item \textbf{Réaction CIA :} Panique~; arrêt de la production~; machines ``corrigées'' (donc sécurisées) redirigées vers des banques suisses.
        \item \textbf{Thèse confirmée :} Une backdoor mathématique ne survit pas à un audit interne compétent~--- ``Security by Obscurity'' est structurellement condamné.
      \end{itemize}
    }
\end{frame}

\begin{frame}{2020 : La Déclassification et le Washington Post}
    \begin{columns}
        \column{0.55\textwidth}
        \textbf{Le Secret Révélé (\#CRYPTOLEAKS)}
        \begin{itemize}
            \item En Février 2020, une enquête conjointe du \textit{Washington Post} (USA), de la \textit{ZDF} (ALL) et de \textit{SRF} (SUI) dévoile l'entièreté du scandale.
            \item Ils publient des documents de la CIA déclassifiés, prouvant que de 1970 à 1993, la quasi-totalité des communications sécurisées mondiales étaient lues par la NSA.
        \end{itemize}

        \vspace{0.3cm}
        \textbf{Le Coup de Maître}
        \begin{itemize}
            \item Un rapport interne de la CIA décrit l'opération Rubicon comme "Le coup de maître du renseignement du siècle".
        \end{itemize}

        \column{0.45\textwidth}
        \begin{center}
            \includegraphics[height=3.5cm]{images/hagelin.jpg}\\
            \vspace{0.1cm}
            \small \textit{Machine classique Hagelin réputée inviolable.}
        \end{center}
    \end{columns}

    \begin{tikzpicture}[remember picture,overlay]
        \node[anchor=south east, xshift=-0.5cm, yshift=0.5cm, font=\tiny, text=gray!50] at (current page.south east) {[Vidéo : Sylvqin, L'Affaire Crypto AG, 2021]};
    \end{tikzpicture}

    \note{
      \begin{itemize}
        \item \textbf{Preuve formelle :} Février 2020~; enquête Washington Post / ZDF / SRF (\#CRYPTOLEAKS).
        \item \textbf{Documents CIA :} Déclassifiés~; prouvent contrôle de +40\% des flux cryptés mondiaux (1970--1993).
        \item \textbf{Citation CIA :} ``Coup de maître du renseignement du siècle''.
        \item \textbf{Liquidation :} Crypto AG dissoute en 2018.
        \item \textbf{Cause racine :} Décalage mathématique subtil sur un polynôme de LFSR.
      \end{itemize}
    }
\end{frame}

% ==============================================================================
% 6. REFERENCES (SECOND TO LAST SLIDE)
% ==============================================================================
\section{Références}

\begin{frame}{Références Bibliographiques}
    \begin{itemize}
        \item \textbf{Support de Cours :}
        \begin{itemize}
            \item G. Castagnos, \textit{Cours de Cryptologie 2025-2026}, Univ. Bordeaux.
            \item Chap. II, Sect. 3 : Chiffrements par flot, LFSR (Def.~II-4, Prop.~II-5), Berlekamp-Massey (Def.~II-8 et Prop.~II-9), immunité de corrélation.
        \end{itemize}

        \vspace{0.3cm}
        \item \textbf{Sources Historiques \& Techniques :}
        \begin{itemize}
            \item G. Miller, \textit{``The intelligence coup of the century''}, Washington Post, 2020.
            \item T. Pornin, \textit{``The Swiss Cheese of Cryptography''}, SSTIC.
            \item J. Gressel \& CCC, Reverse-Engineering HC-7000, \textit{\#CRYPTOLEAKS}, 2020.
            \item Sylvqin, \textit{``L'Affaire Crypto AG~: la plus grande opération d'espionnage du siècle''}, YouTube, 2021.
        \end{itemize}
    \end{itemize}
    \note{
      \begin{itemize}
        \item \textbf{Fondements maths :} Chap.~II (Castagnos)~; Def.~II-4, Prop.~II-5, Def.~II-8, Prop.~II-9.
        \item \textbf{Source primaire :} Documents CIA déclassifiés (Washington Post, 2020).
        \item \textbf{Technique :} Reverse-engineering CCC (Leipzig, 2020).
      \end{itemize}
    }
\end{frame}

% ==============================================================================
% 7. CONCLUSION
% ==============================================================================
\section{Conclusion}

\begin{frame}
    \centering
    \Huge \textbf{Questions ?}

    \vspace{1cm}
    \normalsize
    \textit{``Trust, but Verify.''}\\
    \vspace{0.2cm}
    \small -- \textbf{Ronald Reagan}, sommets nucléaires (1987)
    
    \vspace{0.5cm}
    \normalsize
    \textit{Ou plutôt : ``Don't trust. Open-Source everything.''}

    \note{
      \begin{itemize}
        \item \textbf{Preuve définitive :} ``Security by Obscurity'' est un échec.
        \item \textbf{Risque matériel :} Hardware propriétaire non-audité $=$ backdoor garantie.
        \item \textbf{La leçon :} Seuls les algorithmes publics (AES, NIST) audités résistent.
        \item \textbf{Clôture :} Merci~; questions sur LFSR, corrélation ou contexte géopolitique bienvenues.
      \end{itemize}
    }
\end{frame}

\end{document}