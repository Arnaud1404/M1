\documentclass[aspectratio=169]{beamer}

% Thème Madrid
\usetheme{Madrid}           
\usecolortheme{beaver}

% Packages
\usepackage[utf8]{inputenc}
\usepackage[T1]{fontenc}
\usepackage{booktabs}
\usepackage{listings}
\usepackage{xcolor}
\usepackage{amsmath}
\usepackage{amssymb}
\usepackage{tikz} % Indispensable pour le positionnement du logo

% --- CONFIGURATION DU LOGO (BAS À DROITE) ---
% On utilise le template 'footline' pour placer le logo à la fin de la construction de la slide
\addtobeamertemplate{footline}{}{%
    \begin{tikzpicture}[remember picture,overlay]
        % anchor=south east : on ancre le coin bas-droit du logo
        % at (current page.south east) : on le place au coin bas-droit de la page
        % yshift=0.6cm : on le remonte un peu pour ne pas écraser le numéro de page
        % xshift=-0.2cm : petite marge à droite
        \node[anchor=south east, yshift=0.6cm, xshift=-0.2cm] at (current page.south east) {
            \includegraphics[height=0.8cm]{logo_univ_bdx.png}
        };
    \end{tikzpicture}
}
% -----------------------------

% Configuration pour le code Python (couleurs)
\definecolor{codegray}{rgb}{0.5,0.5,0.5}
\definecolor{codepurple}{rgb}{0.58,0,0.82}
\definecolor{backcolour}{rgb}{0.95,0.95,0.92}

\lstdefinestyle{mystyle}{
    backgroundcolor=\color{backcolour},   
    commentstyle=\color{codegray},
    keywordstyle=\color{magenta},
    numberstyle=\tiny\color{codegray},
    stringstyle=\color{codepurple},
    basicstyle=\ttfamily\footnotesize,
    breakatwhitespace=false,         
    breaklines=true,                 
    captionpos=b,                    
    keepspaces=true,                 
    numbers=left,                    
    numbersep=5pt,                  
    showspaces=false,                
    showstringspaces=false,
    showtabs=false,                  
    tabsize=2
}
\lstset{style=mystyle}

% Infos de la présentation
\title{Operation Rubicon}
\subtitle{Anatomie de la vulnérabilité Minerva}
\date{\today}
\author{Arnaud Gomes}
\institute{Université de Bordeaux} 
% Optionnel : Logo en grand sur la slide de titre aussi
% \titlegraphic{\hfill\includegraphics[height=1.5cm]{logo_univ_bdx.png}}

\begin{document}

\maketitle

% SLIDE 1: INTRO
\begin{frame}{Le Pitch}
    \begin{alertblock}{La plus grande Supply Chain Attack du 20e siècle}
        Pendant 50 ans, la \textbf{CIA} et le \textbf{BND} ont vendu des machines de chiffrement truquées à plus de 100 pays.
    \end{alertblock}
    
    \vspace{0.5cm}
    \begin{itemize}
        \item \textbf{La Cible :} Crypto AG (Suisse).
        \item \textbf{L'Arme :} Pas de micro, pas de bug logiciel... mais des \textbf{Mathématiques}.
        \item \textbf{Le Vecteur :} Hardware "Trusted" (Séries HC-500).
    \end{itemize}
\end{frame}

% SLIDE 2: L'ILLUSION
\begin{frame}{L'Architecture (Le Leurre)}
    \textbf{Le système : Chiffrement par flot (Stream Cipher)}
    
    $$ c_t = m_t \oplus z_t $$
    
    Génération de la suite chiffrante via \textbf{LFSR} (Linear Feedback Shift Registers) :
    
    \begin{itemize}
        \item Polynôme de rétroaction sur $\mathbb{F}_2[X]$ :
        $$ f(X) = 1 \oplus c_1 X \oplus \dots \oplus c_l X^l $$
        \item \textbf{Argument commercial :} Utilisation de polynômes primitifs.
        \item \textbf{Résultat :} Période maximale $T = 2^l - 1$ (m-suite) et équilibre statistique parfait.
    \end{itemize}
    
    \begin{center}
        \textit{"Indistinguable du bruit aléatoire pour les tests statistiques."}
    \end{center}
\end{frame}

% SLIDE 3: LA BACKDOOR
\begin{frame}{L'Exploit : Minerva}
    \begin{columns}
        \column{0.5\textwidth}
        \textbf{Le problème du LFSR pur :}
        Vulnérable à une attaque algébrique en $\mathcal{O}(l^3)$ via Berlekamp-Massey.
        
        \vspace{0.5cm}
        \textbf{La solution (et la faille) :}
        Combinaison non-linéaire filtrée.
        
        $$ z_t = g(S^{(t)}) $$
        
        \column{0.5\textwidth}
        \begin{alertblock}{La Backdoor Statistique}
            La fonction $g$ est truquée pour introduire une \textbf{corrélation} vers un registre cible $L_1$.
            
            $$ P(z_t = L_{1,t}) = 0.5 + \epsilon $$
        \end{alertblock}
    \end{columns}
\end{frame}

% SLIDE 4: DIVIDE AND CONQUER
\begin{frame}{L'Attaque : Divide \& Conquer}
    L'espace de recherche s'effondre.
    
    \begin{itemize}
        \item \textbf{Complexité théorique (Force Brute) :}
        $$ \mathcal{O}(2^{L_1 + L_2 + \dots + L_k}) \approx \infty $$
        
        \item \textbf{Complexité réelle (Minerva) :}
        $$ \mathcal{O}(2^{L_1} + 2^{L_2} + \dots + 2^{L_k}) \approx \text{Quelques secondes} $$
    \end{itemize}
    
    \vspace{0.5cm}
    \textbf{Méthode :}
    \begin{enumerate}
        \item Isoler $L_1$ grâce au biais $\epsilon$.
        \item Reconstruire $L_1$.
        \item Soustraire le flux de $L_1$ et attaquer $L_2$.
    \end{enumerate}
\end{frame}

% SLIDE 5: DEMO CODE
\begin{frame}[fragile]{Preuve de Concept (Python)}
    Simulation du biais de corrélation introduit dans la fonction de combinaison :
    
\begin{lstlisting}[language=Python]
def compromised_combiner(l1, l2, l3):
    # Backdoor: Si l2 & l3 sont nuls, on inverse. 
    # Sinon, on sort l1 tel quel.
    # Resultat : l1 fuite a 75% (Bias epsilon = 0.25)
    
    bias_condition = (l2 | l3)
    if bias_condition == 0:
        return 1 ^ l1
    return l1
\end{lstlisting}
    
    \small{Avec $\epsilon = 0.25$, quelques centaines de bits de chiffré suffisent pour converger.}
\end{frame}

% SLIDE 6: CONCLUSION
\begin{frame}{Conclusion}
    \begin{center}
        \Large \textbf{Hardware is hard... but Math is harder.}
    \end{center}
    
    \vspace{1cm}
    
    \begin{itemize}
        \item La sécurité par l'obscurité est une dette technique.
        \item Le matériel fonctionnait "correctement", c'est la spécification qui était hostile.
        \item \textbf{Takeaway :} Vérifiez vos primitives cryptographiques (Kerckhoffs's principle).
    \end{itemize}
\end{frame}

\begin{frame}[standout]
    Questions ?
\end{frame}

\end{document}